
% Default to the notebook output style

    


% Inherit from the specified cell style.




    
\documentclass[11pt]{article}

    
    
    \usepackage[T1]{fontenc}
    % Nicer default font (+ math font) than Computer Modern for most use cases
    \usepackage{mathpazo}

    % Basic figure setup, for now with no caption control since it's done
    % automatically by Pandoc (which extracts ![](path) syntax from Markdown).
    \usepackage{graphicx}
    % We will generate all images so they have a width \maxwidth. This means
    % that they will get their normal width if they fit onto the page, but
    % are scaled down if they would overflow the margins.
    \makeatletter
    \def\maxwidth{\ifdim\Gin@nat@width>\linewidth\linewidth
    \else\Gin@nat@width\fi}
    \makeatother
    \let\Oldincludegraphics\includegraphics
    % Set max figure width to be 80% of text width, for now hardcoded.
    \renewcommand{\includegraphics}[1]{\Oldincludegraphics[width=.8\maxwidth]{#1}}
    % Ensure that by default, figures have no caption (until we provide a
    % proper Figure object with a Caption API and a way to capture that
    % in the conversion process - todo).
    \usepackage{caption}
    \DeclareCaptionLabelFormat{nolabel}{}
    \captionsetup{labelformat=nolabel}

    \usepackage{adjustbox} % Used to constrain images to a maximum size 
    \usepackage{xcolor} % Allow colors to be defined
    \usepackage{enumerate} % Needed for markdown enumerations to work
    \usepackage{geometry} % Used to adjust the document margins
    \usepackage{amsmath} % Equations
    \usepackage{amssymb} % Equations
    \usepackage{textcomp} % defines textquotesingle
    % Hack from http://tex.stackexchange.com/a/47451/13684:
    \AtBeginDocument{%
        \def\PYZsq{\textquotesingle}% Upright quotes in Pygmentized code
    }
    \usepackage{upquote} % Upright quotes for verbatim code
    \usepackage{eurosym} % defines \euro
    \usepackage[mathletters]{ucs} % Extended unicode (utf-8) support
    \usepackage[utf8x]{inputenc} % Allow utf-8 characters in the tex document
    \usepackage{fancyvrb} % verbatim replacement that allows latex
    \usepackage{grffile} % extends the file name processing of package graphics 
                         % to support a larger range 
    % The hyperref package gives us a pdf with properly built
    % internal navigation ('pdf bookmarks' for the table of contents,
    % internal cross-reference links, web links for URLs, etc.)
    \usepackage{hyperref}
    \usepackage{longtable} % longtable support required by pandoc >1.10
    \usepackage{booktabs}  % table support for pandoc > 1.12.2
    \usepackage[inline]{enumitem} % IRkernel/repr support (it uses the enumerate* environment)
    \usepackage[normalem]{ulem} % ulem is needed to support strikethroughs (\sout)
                                % normalem makes italics be italics, not underlines
    

    
    
    % Colors for the hyperref package
    \definecolor{urlcolor}{rgb}{0,.145,.698}
    \definecolor{linkcolor}{rgb}{.71,0.21,0.01}
    \definecolor{citecolor}{rgb}{.12,.54,.11}

    % ANSI colors
    \definecolor{ansi-black}{HTML}{3E424D}
    \definecolor{ansi-black-intense}{HTML}{282C36}
    \definecolor{ansi-red}{HTML}{E75C58}
    \definecolor{ansi-red-intense}{HTML}{B22B31}
    \definecolor{ansi-green}{HTML}{00A250}
    \definecolor{ansi-green-intense}{HTML}{007427}
    \definecolor{ansi-yellow}{HTML}{DDB62B}
    \definecolor{ansi-yellow-intense}{HTML}{B27D12}
    \definecolor{ansi-blue}{HTML}{208FFB}
    \definecolor{ansi-blue-intense}{HTML}{0065CA}
    \definecolor{ansi-magenta}{HTML}{D160C4}
    \definecolor{ansi-magenta-intense}{HTML}{A03196}
    \definecolor{ansi-cyan}{HTML}{60C6C8}
    \definecolor{ansi-cyan-intense}{HTML}{258F8F}
    \definecolor{ansi-white}{HTML}{C5C1B4}
    \definecolor{ansi-white-intense}{HTML}{A1A6B2}

    % commands and environments needed by pandoc snippets
    % extracted from the output of `pandoc -s`
    \providecommand{\tightlist}{%
      \setlength{\itemsep}{0pt}\setlength{\parskip}{0pt}}
    \DefineVerbatimEnvironment{Highlighting}{Verbatim}{commandchars=\\\{\}}
    % Add ',fontsize=\small' for more characters per line
    \newenvironment{Shaded}{}{}
    \newcommand{\KeywordTok}[1]{\textcolor[rgb]{0.00,0.44,0.13}{\textbf{{#1}}}}
    \newcommand{\DataTypeTok}[1]{\textcolor[rgb]{0.56,0.13,0.00}{{#1}}}
    \newcommand{\DecValTok}[1]{\textcolor[rgb]{0.25,0.63,0.44}{{#1}}}
    \newcommand{\BaseNTok}[1]{\textcolor[rgb]{0.25,0.63,0.44}{{#1}}}
    \newcommand{\FloatTok}[1]{\textcolor[rgb]{0.25,0.63,0.44}{{#1}}}
    \newcommand{\CharTok}[1]{\textcolor[rgb]{0.25,0.44,0.63}{{#1}}}
    \newcommand{\StringTok}[1]{\textcolor[rgb]{0.25,0.44,0.63}{{#1}}}
    \newcommand{\CommentTok}[1]{\textcolor[rgb]{0.38,0.63,0.69}{\textit{{#1}}}}
    \newcommand{\OtherTok}[1]{\textcolor[rgb]{0.00,0.44,0.13}{{#1}}}
    \newcommand{\AlertTok}[1]{\textcolor[rgb]{1.00,0.00,0.00}{\textbf{{#1}}}}
    \newcommand{\FunctionTok}[1]{\textcolor[rgb]{0.02,0.16,0.49}{{#1}}}
    \newcommand{\RegionMarkerTok}[1]{{#1}}
    \newcommand{\ErrorTok}[1]{\textcolor[rgb]{1.00,0.00,0.00}{\textbf{{#1}}}}
    \newcommand{\NormalTok}[1]{{#1}}
    
    % Additional commands for more recent versions of Pandoc
    \newcommand{\ConstantTok}[1]{\textcolor[rgb]{0.53,0.00,0.00}{{#1}}}
    \newcommand{\SpecialCharTok}[1]{\textcolor[rgb]{0.25,0.44,0.63}{{#1}}}
    \newcommand{\VerbatimStringTok}[1]{\textcolor[rgb]{0.25,0.44,0.63}{{#1}}}
    \newcommand{\SpecialStringTok}[1]{\textcolor[rgb]{0.73,0.40,0.53}{{#1}}}
    \newcommand{\ImportTok}[1]{{#1}}
    \newcommand{\DocumentationTok}[1]{\textcolor[rgb]{0.73,0.13,0.13}{\textit{{#1}}}}
    \newcommand{\AnnotationTok}[1]{\textcolor[rgb]{0.38,0.63,0.69}{\textbf{\textit{{#1}}}}}
    \newcommand{\CommentVarTok}[1]{\textcolor[rgb]{0.38,0.63,0.69}{\textbf{\textit{{#1}}}}}
    \newcommand{\VariableTok}[1]{\textcolor[rgb]{0.10,0.09,0.49}{{#1}}}
    \newcommand{\ControlFlowTok}[1]{\textcolor[rgb]{0.00,0.44,0.13}{\textbf{{#1}}}}
    \newcommand{\OperatorTok}[1]{\textcolor[rgb]{0.40,0.40,0.40}{{#1}}}
    \newcommand{\BuiltInTok}[1]{{#1}}
    \newcommand{\ExtensionTok}[1]{{#1}}
    \newcommand{\PreprocessorTok}[1]{\textcolor[rgb]{0.74,0.48,0.00}{{#1}}}
    \newcommand{\AttributeTok}[1]{\textcolor[rgb]{0.49,0.56,0.16}{{#1}}}
    \newcommand{\InformationTok}[1]{\textcolor[rgb]{0.38,0.63,0.69}{\textbf{\textit{{#1}}}}}
    \newcommand{\WarningTok}[1]{\textcolor[rgb]{0.38,0.63,0.69}{\textbf{\textit{{#1}}}}}
    
    
    % Define a nice break command that doesn't care if a line doesn't already
    % exist.
    \def\br{\hspace*{\fill} \\* }
    % Math Jax compatability definitions
    \def\gt{>}
    \def\lt{<}
    % Document parameters
    \title{auswertung}
    
    
    

    % Pygments definitions
    
\makeatletter
\def\PY@reset{\let\PY@it=\relax \let\PY@bf=\relax%
    \let\PY@ul=\relax \let\PY@tc=\relax%
    \let\PY@bc=\relax \let\PY@ff=\relax}
\def\PY@tok#1{\csname PY@tok@#1\endcsname}
\def\PY@toks#1+{\ifx\relax#1\empty\else%
    \PY@tok{#1}\expandafter\PY@toks\fi}
\def\PY@do#1{\PY@bc{\PY@tc{\PY@ul{%
    \PY@it{\PY@bf{\PY@ff{#1}}}}}}}
\def\PY#1#2{\PY@reset\PY@toks#1+\relax+\PY@do{#2}}

\expandafter\def\csname PY@tok@w\endcsname{\def\PY@tc##1{\textcolor[rgb]{0.73,0.73,0.73}{##1}}}
\expandafter\def\csname PY@tok@c\endcsname{\let\PY@it=\textit\def\PY@tc##1{\textcolor[rgb]{0.25,0.50,0.50}{##1}}}
\expandafter\def\csname PY@tok@cp\endcsname{\def\PY@tc##1{\textcolor[rgb]{0.74,0.48,0.00}{##1}}}
\expandafter\def\csname PY@tok@k\endcsname{\let\PY@bf=\textbf\def\PY@tc##1{\textcolor[rgb]{0.00,0.50,0.00}{##1}}}
\expandafter\def\csname PY@tok@kp\endcsname{\def\PY@tc##1{\textcolor[rgb]{0.00,0.50,0.00}{##1}}}
\expandafter\def\csname PY@tok@kt\endcsname{\def\PY@tc##1{\textcolor[rgb]{0.69,0.00,0.25}{##1}}}
\expandafter\def\csname PY@tok@o\endcsname{\def\PY@tc##1{\textcolor[rgb]{0.40,0.40,0.40}{##1}}}
\expandafter\def\csname PY@tok@ow\endcsname{\let\PY@bf=\textbf\def\PY@tc##1{\textcolor[rgb]{0.67,0.13,1.00}{##1}}}
\expandafter\def\csname PY@tok@nb\endcsname{\def\PY@tc##1{\textcolor[rgb]{0.00,0.50,0.00}{##1}}}
\expandafter\def\csname PY@tok@nf\endcsname{\def\PY@tc##1{\textcolor[rgb]{0.00,0.00,1.00}{##1}}}
\expandafter\def\csname PY@tok@nc\endcsname{\let\PY@bf=\textbf\def\PY@tc##1{\textcolor[rgb]{0.00,0.00,1.00}{##1}}}
\expandafter\def\csname PY@tok@nn\endcsname{\let\PY@bf=\textbf\def\PY@tc##1{\textcolor[rgb]{0.00,0.00,1.00}{##1}}}
\expandafter\def\csname PY@tok@ne\endcsname{\let\PY@bf=\textbf\def\PY@tc##1{\textcolor[rgb]{0.82,0.25,0.23}{##1}}}
\expandafter\def\csname PY@tok@nv\endcsname{\def\PY@tc##1{\textcolor[rgb]{0.10,0.09,0.49}{##1}}}
\expandafter\def\csname PY@tok@no\endcsname{\def\PY@tc##1{\textcolor[rgb]{0.53,0.00,0.00}{##1}}}
\expandafter\def\csname PY@tok@nl\endcsname{\def\PY@tc##1{\textcolor[rgb]{0.63,0.63,0.00}{##1}}}
\expandafter\def\csname PY@tok@ni\endcsname{\let\PY@bf=\textbf\def\PY@tc##1{\textcolor[rgb]{0.60,0.60,0.60}{##1}}}
\expandafter\def\csname PY@tok@na\endcsname{\def\PY@tc##1{\textcolor[rgb]{0.49,0.56,0.16}{##1}}}
\expandafter\def\csname PY@tok@nt\endcsname{\let\PY@bf=\textbf\def\PY@tc##1{\textcolor[rgb]{0.00,0.50,0.00}{##1}}}
\expandafter\def\csname PY@tok@nd\endcsname{\def\PY@tc##1{\textcolor[rgb]{0.67,0.13,1.00}{##1}}}
\expandafter\def\csname PY@tok@s\endcsname{\def\PY@tc##1{\textcolor[rgb]{0.73,0.13,0.13}{##1}}}
\expandafter\def\csname PY@tok@sd\endcsname{\let\PY@it=\textit\def\PY@tc##1{\textcolor[rgb]{0.73,0.13,0.13}{##1}}}
\expandafter\def\csname PY@tok@si\endcsname{\let\PY@bf=\textbf\def\PY@tc##1{\textcolor[rgb]{0.73,0.40,0.53}{##1}}}
\expandafter\def\csname PY@tok@se\endcsname{\let\PY@bf=\textbf\def\PY@tc##1{\textcolor[rgb]{0.73,0.40,0.13}{##1}}}
\expandafter\def\csname PY@tok@sr\endcsname{\def\PY@tc##1{\textcolor[rgb]{0.73,0.40,0.53}{##1}}}
\expandafter\def\csname PY@tok@ss\endcsname{\def\PY@tc##1{\textcolor[rgb]{0.10,0.09,0.49}{##1}}}
\expandafter\def\csname PY@tok@sx\endcsname{\def\PY@tc##1{\textcolor[rgb]{0.00,0.50,0.00}{##1}}}
\expandafter\def\csname PY@tok@m\endcsname{\def\PY@tc##1{\textcolor[rgb]{0.40,0.40,0.40}{##1}}}
\expandafter\def\csname PY@tok@gh\endcsname{\let\PY@bf=\textbf\def\PY@tc##1{\textcolor[rgb]{0.00,0.00,0.50}{##1}}}
\expandafter\def\csname PY@tok@gu\endcsname{\let\PY@bf=\textbf\def\PY@tc##1{\textcolor[rgb]{0.50,0.00,0.50}{##1}}}
\expandafter\def\csname PY@tok@gd\endcsname{\def\PY@tc##1{\textcolor[rgb]{0.63,0.00,0.00}{##1}}}
\expandafter\def\csname PY@tok@gi\endcsname{\def\PY@tc##1{\textcolor[rgb]{0.00,0.63,0.00}{##1}}}
\expandafter\def\csname PY@tok@gr\endcsname{\def\PY@tc##1{\textcolor[rgb]{1.00,0.00,0.00}{##1}}}
\expandafter\def\csname PY@tok@ge\endcsname{\let\PY@it=\textit}
\expandafter\def\csname PY@tok@gs\endcsname{\let\PY@bf=\textbf}
\expandafter\def\csname PY@tok@gp\endcsname{\let\PY@bf=\textbf\def\PY@tc##1{\textcolor[rgb]{0.00,0.00,0.50}{##1}}}
\expandafter\def\csname PY@tok@go\endcsname{\def\PY@tc##1{\textcolor[rgb]{0.53,0.53,0.53}{##1}}}
\expandafter\def\csname PY@tok@gt\endcsname{\def\PY@tc##1{\textcolor[rgb]{0.00,0.27,0.87}{##1}}}
\expandafter\def\csname PY@tok@err\endcsname{\def\PY@bc##1{\setlength{\fboxsep}{0pt}\fcolorbox[rgb]{1.00,0.00,0.00}{1,1,1}{\strut ##1}}}
\expandafter\def\csname PY@tok@kc\endcsname{\let\PY@bf=\textbf\def\PY@tc##1{\textcolor[rgb]{0.00,0.50,0.00}{##1}}}
\expandafter\def\csname PY@tok@kd\endcsname{\let\PY@bf=\textbf\def\PY@tc##1{\textcolor[rgb]{0.00,0.50,0.00}{##1}}}
\expandafter\def\csname PY@tok@kn\endcsname{\let\PY@bf=\textbf\def\PY@tc##1{\textcolor[rgb]{0.00,0.50,0.00}{##1}}}
\expandafter\def\csname PY@tok@kr\endcsname{\let\PY@bf=\textbf\def\PY@tc##1{\textcolor[rgb]{0.00,0.50,0.00}{##1}}}
\expandafter\def\csname PY@tok@bp\endcsname{\def\PY@tc##1{\textcolor[rgb]{0.00,0.50,0.00}{##1}}}
\expandafter\def\csname PY@tok@fm\endcsname{\def\PY@tc##1{\textcolor[rgb]{0.00,0.00,1.00}{##1}}}
\expandafter\def\csname PY@tok@vc\endcsname{\def\PY@tc##1{\textcolor[rgb]{0.10,0.09,0.49}{##1}}}
\expandafter\def\csname PY@tok@vg\endcsname{\def\PY@tc##1{\textcolor[rgb]{0.10,0.09,0.49}{##1}}}
\expandafter\def\csname PY@tok@vi\endcsname{\def\PY@tc##1{\textcolor[rgb]{0.10,0.09,0.49}{##1}}}
\expandafter\def\csname PY@tok@vm\endcsname{\def\PY@tc##1{\textcolor[rgb]{0.10,0.09,0.49}{##1}}}
\expandafter\def\csname PY@tok@sa\endcsname{\def\PY@tc##1{\textcolor[rgb]{0.73,0.13,0.13}{##1}}}
\expandafter\def\csname PY@tok@sb\endcsname{\def\PY@tc##1{\textcolor[rgb]{0.73,0.13,0.13}{##1}}}
\expandafter\def\csname PY@tok@sc\endcsname{\def\PY@tc##1{\textcolor[rgb]{0.73,0.13,0.13}{##1}}}
\expandafter\def\csname PY@tok@dl\endcsname{\def\PY@tc##1{\textcolor[rgb]{0.73,0.13,0.13}{##1}}}
\expandafter\def\csname PY@tok@s2\endcsname{\def\PY@tc##1{\textcolor[rgb]{0.73,0.13,0.13}{##1}}}
\expandafter\def\csname PY@tok@sh\endcsname{\def\PY@tc##1{\textcolor[rgb]{0.73,0.13,0.13}{##1}}}
\expandafter\def\csname PY@tok@s1\endcsname{\def\PY@tc##1{\textcolor[rgb]{0.73,0.13,0.13}{##1}}}
\expandafter\def\csname PY@tok@mb\endcsname{\def\PY@tc##1{\textcolor[rgb]{0.40,0.40,0.40}{##1}}}
\expandafter\def\csname PY@tok@mf\endcsname{\def\PY@tc##1{\textcolor[rgb]{0.40,0.40,0.40}{##1}}}
\expandafter\def\csname PY@tok@mh\endcsname{\def\PY@tc##1{\textcolor[rgb]{0.40,0.40,0.40}{##1}}}
\expandafter\def\csname PY@tok@mi\endcsname{\def\PY@tc##1{\textcolor[rgb]{0.40,0.40,0.40}{##1}}}
\expandafter\def\csname PY@tok@il\endcsname{\def\PY@tc##1{\textcolor[rgb]{0.40,0.40,0.40}{##1}}}
\expandafter\def\csname PY@tok@mo\endcsname{\def\PY@tc##1{\textcolor[rgb]{0.40,0.40,0.40}{##1}}}
\expandafter\def\csname PY@tok@ch\endcsname{\let\PY@it=\textit\def\PY@tc##1{\textcolor[rgb]{0.25,0.50,0.50}{##1}}}
\expandafter\def\csname PY@tok@cm\endcsname{\let\PY@it=\textit\def\PY@tc##1{\textcolor[rgb]{0.25,0.50,0.50}{##1}}}
\expandafter\def\csname PY@tok@cpf\endcsname{\let\PY@it=\textit\def\PY@tc##1{\textcolor[rgb]{0.25,0.50,0.50}{##1}}}
\expandafter\def\csname PY@tok@c1\endcsname{\let\PY@it=\textit\def\PY@tc##1{\textcolor[rgb]{0.25,0.50,0.50}{##1}}}
\expandafter\def\csname PY@tok@cs\endcsname{\let\PY@it=\textit\def\PY@tc##1{\textcolor[rgb]{0.25,0.50,0.50}{##1}}}

\def\PYZbs{\char`\\}
\def\PYZus{\char`\_}
\def\PYZob{\char`\{}
\def\PYZcb{\char`\}}
\def\PYZca{\char`\^}
\def\PYZam{\char`\&}
\def\PYZlt{\char`\<}
\def\PYZgt{\char`\>}
\def\PYZsh{\char`\#}
\def\PYZpc{\char`\%}
\def\PYZdl{\char`\$}
\def\PYZhy{\char`\-}
\def\PYZsq{\char`\'}
\def\PYZdq{\char`\"}
\def\PYZti{\char`\~}
% for compatibility with earlier versions
\def\PYZat{@}
\def\PYZlb{[}
\def\PYZrb{]}
\makeatother


    % Exact colors from NB
    \definecolor{incolor}{rgb}{0.0, 0.0, 0.5}
    \definecolor{outcolor}{rgb}{0.545, 0.0, 0.0}



    
    % Prevent overflowing lines due to hard-to-break entities
    \sloppy 
    % Setup hyperref package
    \hypersetup{
      breaklinks=true,  % so long urls are correctly broken across lines
      colorlinks=true,
      urlcolor=urlcolor,
      linkcolor=linkcolor,
      citecolor=citecolor,
      }
    % Slightly bigger margins than the latex defaults
    
    \geometry{verbose,tmargin=1in,bmargin=1in,lmargin=1in,rmargin=1in}
    
    

    \begin{document}
    
    
    \maketitle
    
    

    
    \begin{Verbatim}[commandchars=\\\{\}]
{\color{incolor}In [{\color{incolor}1}]:} \PY{o}{\PYZpc{}}\PY{k}{matplotlib} inline
        \PY{k+kn}{import} \PY{n+nn}{matplotlib}\PY{n+nn}{.}\PY{n+nn}{pyplot} \PY{k}{as} \PY{n+nn}{plt}
        \PY{k+kn}{import} \PY{n+nn}{numpy} \PY{k}{as} \PY{n+nn}{np}
        \PY{k+kn}{from} \PY{n+nn}{scipy}\PY{n+nn}{.}\PY{n+nn}{optimize} \PY{k}{import} \PY{n}{curve\PYZus{}fit}
        \PY{k+kn}{from} \PY{n+nn}{scipy}\PY{n+nn}{.}\PY{n+nn}{stats} \PY{k}{import} \PY{n}{chi2}
        \PY{k+kn}{from} \PY{n+nn}{math} \PY{k}{import} \PY{n}{sqrt}
\end{Verbatim}


    SpectraSuite verwendet Kommata als Dezimaltrennzeichen, die durch Punkte
ersetzt werden müssen:

    \begin{Verbatim}[commandchars=\\\{\}]
{\color{incolor}In [{\color{incolor}2}]:} \PY{k}{def} \PY{n+nf}{comma\PYZus{}to\PYZus{}float}\PY{p}{(}\PY{n}{valstr}\PY{p}{)}\PY{p}{:}
            \PY{k}{return} \PY{n+nb}{float}\PY{p}{(}\PY{n}{valstr}\PY{o}{.}\PY{n}{decode}\PY{p}{(}\PY{l+s+s2}{\PYZdq{}}\PY{l+s+s2}{utf\PYZhy{}8}\PY{l+s+s2}{\PYZdq{}}\PY{p}{)}\PY{o}{.}\PY{n}{replace}\PY{p}{(}\PY{l+s+s1}{\PYZsq{}}\PY{l+s+s1}{,}\PY{l+s+s1}{\PYZsq{}}\PY{p}{,} \PY{l+s+s1}{\PYZsq{}}\PY{l+s+s1}{.}\PY{l+s+s1}{\PYZsq{}}\PY{p}{)}\PY{p}{)}
\end{Verbatim}


    \hypertarget{analyse-des-sonnenlichts}{%
\section{Analyse des Sonnenlichts}\label{analyse-des-sonnenlichts}}

    Zunächst werden die Daten mit und ohne Fenster geladen. Die ersten 17
Zeilen sowie alle Zeilen, die mit `\textgreater{}' beginnen, sind Header
und Kommentare und werden deshalb ausgelassen.

    \begin{Verbatim}[commandchars=\\\{\}]
{\color{incolor}In [{\color{incolor}3}]:} \PY{n}{lamb\PYZus{}og}\PY{p}{,} \PY{n}{inten\PYZus{}og} \PY{o}{=} \PY{n}{np}\PY{o}{.}\PY{n}{loadtxt}\PY{p}{(}\PY{l+s+s1}{\PYZsq{}}\PY{l+s+s1}{data/sonne\PYZus{}ohne\PYZus{}glas.txt}\PY{l+s+s1}{\PYZsq{}}\PY{p}{,}
                                       \PY{n}{skiprows} \PY{o}{=} \PY{l+m+mi}{17}\PY{p}{,}
                                       \PY{n}{converters} \PY{o}{=} \PY{p}{\PYZob{}}\PY{l+m+mi}{0}\PY{p}{:}\PY{n}{comma\PYZus{}to\PYZus{}float}\PY{p}{,} \PY{l+m+mi}{1}\PY{p}{:}\PY{n}{comma\PYZus{}to\PYZus{}float}\PY{p}{\PYZcb{}}\PY{p}{,}
                                       \PY{n}{comments} \PY{o}{=} \PY{l+s+s1}{\PYZsq{}}\PY{l+s+s1}{\PYZgt{}}\PY{l+s+s1}{\PYZsq{}}\PY{p}{,}
                                       \PY{n}{unpack} \PY{o}{=} \PY{k+kc}{True}\PY{p}{)}
\end{Verbatim}


    \begin{Verbatim}[commandchars=\\\{\}]
{\color{incolor}In [{\color{incolor}4}]:} \PY{n}{lamb\PYZus{}mg}\PY{p}{,} \PY{n}{inten\PYZus{}mg} \PY{o}{=} \PY{n}{np}\PY{o}{.}\PY{n}{loadtxt}\PY{p}{(}\PY{l+s+s1}{\PYZsq{}}\PY{l+s+s1}{data/sonne\PYZus{}durch\PYZus{}glas.txt}\PY{l+s+s1}{\PYZsq{}}\PY{p}{,}
                                       \PY{n}{skiprows} \PY{o}{=} \PY{l+m+mi}{17}\PY{p}{,}
                                       \PY{n}{converters} \PY{o}{=} \PY{p}{\PYZob{}}\PY{l+m+mi}{0}\PY{p}{:}\PY{n}{comma\PYZus{}to\PYZus{}float}\PY{p}{,} \PY{l+m+mi}{1}\PY{p}{:}\PY{n}{comma\PYZus{}to\PYZus{}float}\PY{p}{\PYZcb{}}\PY{p}{,}
                                       \PY{n}{comments} \PY{o}{=} \PY{l+s+s1}{\PYZsq{}}\PY{l+s+s1}{\PYZgt{}}\PY{l+s+s1}{\PYZsq{}}\PY{p}{,}
                                       \PY{n}{unpack} \PY{o}{=} \PY{k+kc}{True}\PY{p}{)}
\end{Verbatim}


    Die beiden Intensitätsverteilungen werden nun in ein gemeinsames
Diagramm eingezeichnet:

    \begin{Verbatim}[commandchars=\\\{\}]
{\color{incolor}In [{\color{incolor}77}]:} \PY{n}{plt}\PY{o}{.}\PY{n}{plot}\PY{p}{(}\PY{n}{lamb\PYZus{}og}\PY{p}{,} \PY{n}{inten\PYZus{}og}\PY{p}{,} \PY{n}{label}\PY{o}{=}\PY{l+s+s2}{\PYZdq{}}\PY{l+s+s2}{Ohne Fenster}\PY{l+s+s2}{\PYZdq{}}\PY{p}{,} \PY{n}{linewidth} \PY{o}{=} \PY{l+m+mi}{1}\PY{p}{)}
         \PY{n}{plt}\PY{o}{.}\PY{n}{plot}\PY{p}{(}\PY{n}{lamb\PYZus{}mg}\PY{p}{,} \PY{n}{inten\PYZus{}mg}\PY{p}{,} \PY{n}{label}\PY{o}{=}\PY{l+s+s2}{\PYZdq{}}\PY{l+s+s2}{Mit Fenster}\PY{l+s+s2}{\PYZdq{}}\PY{p}{,} \PY{n}{linewidth} \PY{o}{=} \PY{l+m+mi}{1}\PY{p}{)}
         \PY{n}{plt}\PY{o}{.}\PY{n}{title}\PY{p}{(}\PY{l+s+s2}{\PYZdq{}}\PY{l+s+s2}{Gemessenes Sonnenspektrium mit und ohne Fenster}\PY{l+s+s2}{\PYZdq{}}\PY{p}{)}
         \PY{n}{plt}\PY{o}{.}\PY{n}{xlabel}\PY{p}{(}\PY{l+s+s2}{\PYZdq{}}\PY{l+s+s2}{Wellenlänge [nm]}\PY{l+s+s2}{\PYZdq{}}\PY{p}{)}
         \PY{n}{plt}\PY{o}{.}\PY{n}{ylabel}\PY{p}{(}\PY{l+s+s2}{\PYZdq{}}\PY{l+s+s2}{Intensität [b.E.]}\PY{l+s+s2}{\PYZdq{}}\PY{p}{)}
         \PY{n}{plt}\PY{o}{.}\PY{n}{legend}\PY{p}{(}\PY{p}{)}
         \PY{n}{plt}\PY{o}{.}\PY{n}{grid}\PY{p}{(}\PY{p}{)}
         \PY{n}{plt}\PY{o}{.}\PY{n}{ylim}\PY{p}{(}\PY{p}{(}\PY{l+m+mi}{0}\PY{p}{,} \PY{l+m+mi}{65000}\PY{p}{)}\PY{p}{)}
         \PY{n}{plt}\PY{o}{.}\PY{n}{xlim}\PY{p}{(}\PY{p}{(}\PY{l+m+mi}{250}\PY{p}{,} \PY{l+m+mi}{900}\PY{p}{)}\PY{p}{)}
         \PY{n}{plt}\PY{o}{.}\PY{n}{savefig}\PY{p}{(}\PY{l+s+s2}{\PYZdq{}}\PY{l+s+s2}{figures/sonnenlicht\PYZus{}mit\PYZus{}ohne\PYZus{}glas.pdf}\PY{l+s+s2}{\PYZdq{}}\PY{p}{,} \PY{n+nb}{format} \PY{o}{=} \PY{l+s+s2}{\PYZdq{}}\PY{l+s+s2}{pdf}\PY{l+s+s2}{\PYZdq{}}\PY{p}{)}
\end{Verbatim}


    \begin{center}
    \adjustimage{max size={0.9\linewidth}{0.9\paperheight}}{output_8_0.png}
    \end{center}
    { \hspace*{\fill} \\}
    
    \hypertarget{absorption-von-glas}{%
\subsection{Absorption von Glas}\label{absorption-von-glas}}

    Die Absorption von Glas ist gegeben durch
\[A(\lambda) = 1 - \frac{I_{mG}(\lambda)}{I_{oG}(\lambda)}\]

    \begin{Verbatim}[commandchars=\\\{\}]
{\color{incolor}In [{\color{incolor}6}]:} \PY{n}{A} \PY{o}{=} \PY{l+m+mi}{1} \PY{o}{\PYZhy{}} \PY{n}{inten\PYZus{}mg}\PY{o}{/}\PY{n}{inten\PYZus{}og}
\end{Verbatim}


    Die Absorption wird nun als Funktion der Wellenlänge dargestellt:

    \begin{Verbatim}[commandchars=\\\{\}]
{\color{incolor}In [{\color{incolor}7}]:} \PY{n}{plt}\PY{o}{.}\PY{n}{plot}\PY{p}{(}\PY{n}{lamb\PYZus{}mg}\PY{p}{,} \PY{n}{A}\PY{p}{)}
        \PY{n}{plt}\PY{o}{.}\PY{n}{title}\PY{p}{(}\PY{l+s+s2}{\PYZdq{}}\PY{l+s+s2}{Absorption von Glas}\PY{l+s+s2}{\PYZdq{}}\PY{p}{)}
        \PY{n}{plt}\PY{o}{.}\PY{n}{xlabel}\PY{p}{(}\PY{l+s+s2}{\PYZdq{}}\PY{l+s+s2}{Wellenlänge [nm]}\PY{l+s+s2}{\PYZdq{}}\PY{p}{)}
        \PY{n}{plt}\PY{o}{.}\PY{n}{ylabel}\PY{p}{(}\PY{l+s+s1}{\PYZsq{}}\PY{l+s+s1}{Absorption [b.E.]}\PY{l+s+s1}{\PYZsq{}}\PY{p}{)}
        \PY{n}{plt}\PY{o}{.}\PY{n}{ylim}\PY{p}{(}\PY{p}{(}\PY{l+m+mi}{0}\PY{p}{,} \PY{l+m+mi}{1}\PY{p}{)}\PY{p}{)}
        \PY{n}{plt}\PY{o}{.}\PY{n}{xlim}\PY{p}{(}\PY{p}{(}\PY{l+m+mi}{320}\PY{p}{,} \PY{l+m+mi}{800}\PY{p}{)}\PY{p}{)}
        \PY{n}{plt}\PY{o}{.}\PY{n}{savefig}\PY{p}{(}\PY{l+s+s2}{\PYZdq{}}\PY{l+s+s2}{figures/absorption\PYZus{}glas.pdf}\PY{l+s+s2}{\PYZdq{}}\PY{p}{,} \PY{n+nb}{format} \PY{o}{=} \PY{l+s+s2}{\PYZdq{}}\PY{l+s+s2}{pdf}\PY{l+s+s2}{\PYZdq{}}\PY{p}{)}
\end{Verbatim}


    \begin{center}
    \adjustimage{max size={0.9\linewidth}{0.9\paperheight}}{output_13_0.png}
    \end{center}
    { \hspace*{\fill} \\}
    
    \hypertarget{fraunhoferlinien}{%
\subsection{Fraunhoferlinien}\label{fraunhoferlinien}}

    In einem interaktiven Plot werden die Fraunhoferlinien abgelesen:

    \begin{Verbatim}[commandchars=\\\{\}]
{\color{incolor}In [{\color{incolor}8}]:} \PY{o}{\PYZpc{}}\PY{k}{matplotlib} widget
        \PY{n}{plt}\PY{o}{.}\PY{n}{plot}\PY{p}{(}\PY{n}{lamb\PYZus{}og}\PY{p}{,} \PY{n}{inten\PYZus{}og}\PY{p}{)}
        \PY{n}{plt}\PY{o}{.}\PY{n}{title}\PY{p}{(}\PY{l+s+s2}{\PYZdq{}}\PY{l+s+s2}{Sonnenspektrum}\PY{l+s+s2}{\PYZdq{}}\PY{p}{)}
        \PY{n}{plt}\PY{o}{.}\PY{n}{xlabel}\PY{p}{(}\PY{l+s+s2}{\PYZdq{}}\PY{l+s+s2}{Wellenlänge [nm]}\PY{l+s+s2}{\PYZdq{}}\PY{p}{)}
        \PY{n}{plt}\PY{o}{.}\PY{n}{ylabel}\PY{p}{(}\PY{l+s+s2}{\PYZdq{}}\PY{l+s+s2}{Intensität [b.E.]}\PY{l+s+s2}{\PYZdq{}}\PY{p}{)}
        \PY{n}{plt}\PY{o}{.}\PY{n}{ylim}\PY{p}{(}\PY{p}{(}\PY{l+m+mi}{0}\PY{p}{,} \PY{l+m+mi}{65000}\PY{p}{)}\PY{p}{)}
        \PY{n}{plt}\PY{o}{.}\PY{n}{xlim}\PY{p}{(}\PY{p}{(}\PY{l+m+mi}{350}\PY{p}{,} \PY{l+m+mi}{800}\PY{p}{)}\PY{p}{)}
        \PY{n}{plt}\PY{o}{.}\PY{n}{savefig}\PY{p}{(}\PY{l+s+s2}{\PYZdq{}}\PY{l+s+s2}{figures/sonnenspektrum.pdf}\PY{l+s+s2}{\PYZdq{}}\PY{p}{,} \PY{n+nb}{format} \PY{o}{=} \PY{l+s+s2}{\PYZdq{}}\PY{l+s+s2}{pdf}\PY{l+s+s2}{\PYZdq{}}\PY{p}{)}
\end{Verbatim}


    
    \begin{verbatim}
FigureCanvasNbAgg()
    \end{verbatim}

    
    \begin{center}
    \adjustimage{max size={0.9\linewidth}{0.9\paperheight}}{output_16_1.png}
    \end{center}
    { \hspace*{\fill} \\}
    
    Die abgelesenen Wellenlängen werden nun in einer neuen Abbildung
dargestellt:

    \begin{Verbatim}[commandchars=\\\{\}]
{\color{incolor}In [{\color{incolor}76}]:} \PY{o}{\PYZpc{}}\PY{k}{matplotlib} inline
         \PY{n}{plt}\PY{o}{.}\PY{n}{plot}\PY{p}{(}\PY{n}{lamb\PYZus{}og}\PY{p}{,} \PY{n}{inten\PYZus{}og}\PY{p}{,} \PY{n}{linewidth} \PY{o}{=} \PY{l+m+mi}{1}\PY{p}{)}
         \PY{n}{plt}\PY{o}{.}\PY{n}{title}\PY{p}{(}\PY{l+s+s2}{\PYZdq{}}\PY{l+s+s2}{Sonnenspektrum mit Fraunhoferlinien}\PY{l+s+s2}{\PYZdq{}}\PY{p}{)}
         \PY{n}{plt}\PY{o}{.}\PY{n}{xlabel}\PY{p}{(}\PY{l+s+s2}{\PYZdq{}}\PY{l+s+s2}{Wellenlänge [nm]}\PY{l+s+s2}{\PYZdq{}}\PY{p}{)}
         \PY{n}{plt}\PY{o}{.}\PY{n}{ylabel}\PY{p}{(}\PY{l+s+s2}{\PYZdq{}}\PY{l+s+s2}{Intensität [b.E.]}\PY{l+s+s2}{\PYZdq{}}\PY{p}{)}
         \PY{n}{plt}\PY{o}{.}\PY{n}{ylim}\PY{p}{(}\PY{p}{(}\PY{l+m+mi}{0}\PY{p}{,} \PY{l+m+mi}{65000}\PY{p}{)}\PY{p}{)}
         \PY{n}{plt}\PY{o}{.}\PY{n}{xlim}\PY{p}{(}\PY{p}{(}\PY{l+m+mi}{350}\PY{p}{,} \PY{l+m+mi}{800}\PY{p}{)}\PY{p}{)}
         
         \PY{c+c1}{\PYZsh{} abgelesene Werte:}
         \PY{n}{fraunhofer\PYZus{}linien} \PY{o}{=} \PY{p}{[}\PY{l+m+mf}{430.2}\PY{p}{,} \PY{l+m+mf}{485.9}\PY{p}{,} \PY{l+m+mf}{517.3}\PY{p}{,} \PY{l+m+mf}{526.7}\PY{p}{,} \PY{l+m+mf}{589.1}\PY{p}{,} \PY{l+m+mf}{656.0}\PY{p}{,} \PY{l+m+mf}{687.0}\PY{p}{,} \PY{l+m+mf}{719.4}\PY{p}{,} \PY{l+m+mf}{760.3}\PY{p}{]}
         \PY{k}{for} \PY{n}{linie} \PY{o+ow}{in} \PY{n}{fraunhofer\PYZus{}linien}\PY{p}{:}
             \PY{c+c1}{\PYZsh{} vertikale Gerade einzeichnen}
             \PY{n}{plt}\PY{o}{.}\PY{n}{axvline}\PY{p}{(}\PY{n}{x} \PY{o}{=} \PY{n}{linie}\PY{p}{,} \PY{n}{linewidth} \PY{o}{=} \PY{l+m+mi}{1}\PY{p}{,} \PY{n}{color} \PY{o}{=} \PY{l+s+s2}{\PYZdq{}}\PY{l+s+s2}{orange}\PY{l+s+s2}{\PYZdq{}}\PY{p}{)}
         
         \PY{n}{plt}\PY{o}{.}\PY{n}{savefig}\PY{p}{(}\PY{l+s+s2}{\PYZdq{}}\PY{l+s+s2}{figures/fraunhofer.pdf}\PY{l+s+s2}{\PYZdq{}}\PY{p}{,} \PY{n+nb}{format} \PY{o}{=} \PY{l+s+s2}{\PYZdq{}}\PY{l+s+s2}{pdf}\PY{l+s+s2}{\PYZdq{}}\PY{p}{)}
\end{Verbatim}


    \begin{center}
    \adjustimage{max size={0.9\linewidth}{0.9\paperheight}}{output_18_0.png}
    \end{center}
    { \hspace*{\fill} \\}
    
    \hypertarget{natriumspektrum}{%
\section{Natriumspektrum}\label{natriumspektrum}}

    Zunächst werden die Datensätze wie oben geladen:

    \begin{Verbatim}[commandchars=\\\{\}]
{\color{incolor}In [{\color{incolor}10}]:} \PY{c+c1}{\PYZsh{} Integrationszeit so gewählt, dass hellste Linie nicht in Sättigung}
         \PY{n}{lamb\PYZus{}gesamt\PYZus{}1}\PY{p}{,} \PY{n}{inten\PYZus{}gesamt\PYZus{}1} \PY{o}{=} \PY{n}{np}\PY{o}{.}\PY{n}{loadtxt}\PY{p}{(}\PY{l+s+s1}{\PYZsq{}}\PY{l+s+s1}{data/natrium\PYZus{}komplett\PYZus{}1.txt}\PY{l+s+s1}{\PYZsq{}}\PY{p}{,}
                                                    \PY{n}{skiprows} \PY{o}{=} \PY{l+m+mi}{17}\PY{p}{,}
                                                    \PY{n}{converters} \PY{o}{=} \PY{p}{\PYZob{}}\PY{l+m+mi}{0}\PY{p}{:}\PY{n}{comma\PYZus{}to\PYZus{}float}\PY{p}{,} \PY{l+m+mi}{1}\PY{p}{:}\PY{n}{comma\PYZus{}to\PYZus{}float}\PY{p}{\PYZcb{}}\PY{p}{,}
                                                    \PY{n}{comments} \PY{o}{=} \PY{l+s+s1}{\PYZsq{}}\PY{l+s+s1}{\PYZgt{}}\PY{l+s+s1}{\PYZsq{}}\PY{p}{,}
                                                    \PY{n}{unpack} \PY{o}{=} \PY{k+kc}{True}\PY{p}{)}
         
         \PY{c+c1}{\PYZsh{} Längere Integrationszeit, damit wesentlicher Teil des Spektrums sichtbar}
         \PY{n}{lamb\PYZus{}gesamt\PYZus{}2}\PY{p}{,} \PY{n}{inten\PYZus{}gesamt\PYZus{}2} \PY{o}{=} \PY{n}{np}\PY{o}{.}\PY{n}{loadtxt}\PY{p}{(}\PY{l+s+s1}{\PYZsq{}}\PY{l+s+s1}{data/natrium\PYZus{}komplett\PYZus{}2.txt}\PY{l+s+s1}{\PYZsq{}}\PY{p}{,}
                                                    \PY{n}{skiprows} \PY{o}{=} \PY{l+m+mi}{17}\PY{p}{,}
                                                    \PY{n}{converters} \PY{o}{=} \PY{p}{\PYZob{}}\PY{l+m+mi}{0}\PY{p}{:}\PY{n}{comma\PYZus{}to\PYZus{}float}\PY{p}{,} \PY{l+m+mi}{1}\PY{p}{:}\PY{n}{comma\PYZus{}to\PYZus{}float}\PY{p}{\PYZcb{}}\PY{p}{,}
                                                    \PY{n}{comments} \PY{o}{=} \PY{l+s+s1}{\PYZsq{}}\PY{l+s+s1}{\PYZgt{}}\PY{l+s+s1}{\PYZsq{}}\PY{p}{,}
                                                    \PY{n}{unpack} \PY{o}{=} \PY{k+kc}{True}\PY{p}{)}
         
         \PY{c+c1}{\PYZsh{} Integrationszeit für 400 \PYZhy{} 540 nm optimiert}
         \PY{n}{lamb\PYZus{}schwach}\PY{p}{,} \PY{n}{inten\PYZus{}schwach} \PY{o}{=} \PY{n}{np}\PY{o}{.}\PY{n}{loadtxt}\PY{p}{(}\PY{l+s+s1}{\PYZsq{}}\PY{l+s+s1}{data/natrium\PYZus{}niedrige\PYZus{}intensitaet.txt}\PY{l+s+s1}{\PYZsq{}}\PY{p}{,}
                                                  \PY{n}{skiprows} \PY{o}{=} \PY{l+m+mi}{17}\PY{p}{,}
                                                  \PY{n}{converters} \PY{o}{=} \PY{p}{\PYZob{}}\PY{l+m+mi}{0}\PY{p}{:}\PY{n}{comma\PYZus{}to\PYZus{}float}\PY{p}{,} \PY{l+m+mi}{1}\PY{p}{:}\PY{n}{comma\PYZus{}to\PYZus{}float}\PY{p}{\PYZcb{}}\PY{p}{,}
                                                  \PY{n}{comments} \PY{o}{=} \PY{l+s+s1}{\PYZsq{}}\PY{l+s+s1}{\PYZgt{}}\PY{l+s+s1}{\PYZsq{}}\PY{p}{,}
                                                  \PY{n}{unpack} \PY{o}{=} \PY{k+kc}{True}\PY{p}{)}
\end{Verbatim}


    Zunächst wird das gesamte Spektrum geplottet:

    \begin{Verbatim}[commandchars=\\\{\}]
{\color{incolor}In [{\color{incolor}11}]:} \PY{n}{plt}\PY{o}{.}\PY{n}{plot}\PY{p}{(}\PY{n}{lamb\PYZus{}gesamt\PYZus{}1}\PY{p}{,} \PY{n}{inten\PYZus{}gesamt\PYZus{}1}\PY{p}{,} \PY{n}{linewidth} \PY{o}{=} \PY{l+m+mi}{1}\PY{p}{)}
         \PY{n}{plt}\PY{o}{.}\PY{n}{title}\PY{p}{(}\PY{l+s+s2}{\PYZdq{}}\PY{l+s+s2}{Natriumspektrum (gesamt)}\PY{l+s+s2}{\PYZdq{}}\PY{p}{)}
         \PY{n}{plt}\PY{o}{.}\PY{n}{xlabel}\PY{p}{(}\PY{l+s+s2}{\PYZdq{}}\PY{l+s+s2}{Wellenlänge [nm]}\PY{l+s+s2}{\PYZdq{}}\PY{p}{)}
         \PY{n}{plt}\PY{o}{.}\PY{n}{ylabel}\PY{p}{(}\PY{l+s+s2}{\PYZdq{}}\PY{l+s+s2}{Intensität [b.E.]}\PY{l+s+s2}{\PYZdq{}}\PY{p}{)}
         \PY{n}{plt}\PY{o}{.}\PY{n}{yscale}\PY{p}{(}\PY{l+s+s2}{\PYZdq{}}\PY{l+s+s2}{log}\PY{l+s+s2}{\PYZdq{}}\PY{p}{)}
         \PY{n}{plt}\PY{o}{.}\PY{n}{ylim}\PY{p}{(}\PY{p}{(}\PY{l+m+mi}{10}\PY{p}{,} \PY{l+m+mi}{60000}\PY{p}{)}\PY{p}{)}
         \PY{n}{plt}\PY{o}{.}\PY{n}{xlim}\PY{p}{(}\PY{p}{(}\PY{l+m+mi}{350}\PY{p}{,} \PY{l+m+mi}{800}\PY{p}{)}\PY{p}{)}
         \PY{n}{plt}\PY{o}{.}\PY{n}{savefig}\PY{p}{(}\PY{l+s+s2}{\PYZdq{}}\PY{l+s+s2}{figures/natrium\PYZus{}komplett.pdf}\PY{l+s+s2}{\PYZdq{}}\PY{p}{,} \PY{n+nb}{format} \PY{o}{=} \PY{l+s+s2}{\PYZdq{}}\PY{l+s+s2}{pdf}\PY{l+s+s2}{\PYZdq{}}\PY{p}{)}
\end{Verbatim}


    \begin{center}
    \adjustimage{max size={0.9\linewidth}{0.9\paperheight}}{output_23_0.png}
    \end{center}
    { \hspace*{\fill} \\}
    
    Um die Linien im Bereich 600nm - 850nm besser zu erkennen, wird dieser
Abschnitt extra geplottet, wobei der dazu besser geeignete Datensatz
verwendet wird (höhere Integrationszeit).

    \begin{Verbatim}[commandchars=\\\{\}]
{\color{incolor}In [{\color{incolor}12}]:} \PY{n}{plt}\PY{o}{.}\PY{n}{plot}\PY{p}{(}\PY{n}{lamb\PYZus{}gesamt\PYZus{}2}\PY{p}{,} \PY{n}{inten\PYZus{}gesamt\PYZus{}2}\PY{p}{,} \PY{n}{linewidth} \PY{o}{=} \PY{l+m+mi}{1}\PY{p}{)}
         \PY{n}{plt}\PY{o}{.}\PY{n}{title}\PY{p}{(}\PY{l+s+s2}{\PYZdq{}}\PY{l+s+s2}{Natriumspektrum (langwellig)}\PY{l+s+s2}{\PYZdq{}}\PY{p}{)}
         \PY{n}{plt}\PY{o}{.}\PY{n}{xlabel}\PY{p}{(}\PY{l+s+s2}{\PYZdq{}}\PY{l+s+s2}{Wellenlänge [nm]}\PY{l+s+s2}{\PYZdq{}}\PY{p}{)}
         \PY{n}{plt}\PY{o}{.}\PY{n}{ylabel}\PY{p}{(}\PY{l+s+s2}{\PYZdq{}}\PY{l+s+s2}{Intensität [b.E.]}\PY{l+s+s2}{\PYZdq{}}\PY{p}{)}
         \PY{n}{plt}\PY{o}{.}\PY{n}{yscale}\PY{p}{(}\PY{l+s+s2}{\PYZdq{}}\PY{l+s+s2}{log}\PY{l+s+s2}{\PYZdq{}}\PY{p}{)}
         \PY{n}{plt}\PY{o}{.}\PY{n}{ylim}\PY{p}{(}\PY{p}{(}\PY{l+m+mi}{40}\PY{p}{,} \PY{l+m+mi}{60000}\PY{p}{)}\PY{p}{)}
         \PY{n}{plt}\PY{o}{.}\PY{n}{xlim}\PY{p}{(}\PY{p}{(}\PY{l+m+mi}{600}\PY{p}{,} \PY{l+m+mi}{850}\PY{p}{)}\PY{p}{)}
         \PY{n}{plt}\PY{o}{.}\PY{n}{savefig}\PY{p}{(}\PY{l+s+s2}{\PYZdq{}}\PY{l+s+s2}{figures/natrium\PYZus{}langwellig.pdf}\PY{l+s+s2}{\PYZdq{}}\PY{p}{,} \PY{n+nb}{format} \PY{o}{=} \PY{l+s+s2}{\PYZdq{}}\PY{l+s+s2}{pdf}\PY{l+s+s2}{\PYZdq{}}\PY{p}{)}
\end{Verbatim}


    \begin{center}
    \adjustimage{max size={0.9\linewidth}{0.9\paperheight}}{output_25_0.png}
    \end{center}
    { \hspace*{\fill} \\}
    
    Schließlich wird der Bereich von 300nm bis 540nm mit dem dafür
optimierten Datensatz geplottet.

    \begin{Verbatim}[commandchars=\\\{\}]
{\color{incolor}In [{\color{incolor}13}]:} \PY{n}{plt}\PY{o}{.}\PY{n}{plot}\PY{p}{(}\PY{n}{lamb\PYZus{}schwach}\PY{p}{,} \PY{n}{inten\PYZus{}schwach}\PY{p}{,} \PY{n}{linewidth} \PY{o}{=} \PY{l+m+mi}{1}\PY{p}{)}
         \PY{n}{plt}\PY{o}{.}\PY{n}{title}\PY{p}{(}\PY{l+s+s2}{\PYZdq{}}\PY{l+s+s2}{Natriumspektrum (kurzwellig)}\PY{l+s+s2}{\PYZdq{}}\PY{p}{)}
         \PY{n}{plt}\PY{o}{.}\PY{n}{xlabel}\PY{p}{(}\PY{l+s+s2}{\PYZdq{}}\PY{l+s+s2}{Wellenlänge [nm]}\PY{l+s+s2}{\PYZdq{}}\PY{p}{)}
         \PY{n}{plt}\PY{o}{.}\PY{n}{ylabel}\PY{p}{(}\PY{l+s+s2}{\PYZdq{}}\PY{l+s+s2}{Intensität [b.E.]}\PY{l+s+s2}{\PYZdq{}}\PY{p}{)}
         \PY{n}{plt}\PY{o}{.}\PY{n}{yscale}\PY{p}{(}\PY{l+s+s2}{\PYZdq{}}\PY{l+s+s2}{log}\PY{l+s+s2}{\PYZdq{}}\PY{p}{)}
         \PY{n}{plt}\PY{o}{.}\PY{n}{ylim}\PY{p}{(}\PY{p}{(}\PY{l+m+mi}{250}\PY{p}{,} \PY{l+m+mi}{60000}\PY{p}{)}\PY{p}{)}
         \PY{n}{plt}\PY{o}{.}\PY{n}{xlim}\PY{p}{(}\PY{p}{(}\PY{l+m+mi}{300}\PY{p}{,} \PY{l+m+mi}{540}\PY{p}{)}\PY{p}{)}
         \PY{n}{plt}\PY{o}{.}\PY{n}{savefig}\PY{p}{(}\PY{l+s+s2}{\PYZdq{}}\PY{l+s+s2}{figures/natrium\PYZus{}schwach.pdf}\PY{l+s+s2}{\PYZdq{}}\PY{p}{,} \PY{n+nb}{format} \PY{o}{=} \PY{l+s+s2}{\PYZdq{}}\PY{l+s+s2}{pdf}\PY{l+s+s2}{\PYZdq{}}\PY{p}{)}
\end{Verbatim}


    \begin{center}
    \adjustimage{max size={0.9\linewidth}{0.9\paperheight}}{output_27_0.png}
    \end{center}
    { \hspace*{\fill} \\}
    
    \hypertarget{zuordnung-der-gefundenen-linien-zu-serien}{%
\subsection{Zuordnung der gefundenen Linien zu
Serien}\label{zuordnung-der-gefundenen-linien-zu-serien}}

    Einige benötigte Naturkonstanten:

    \begin{Verbatim}[commandchars=\\\{\}]
{\color{incolor}In [{\color{incolor}14}]:} \PY{n}{E\PYZus{}Ry} \PY{o}{=} \PY{o}{\PYZhy{}}\PY{l+m+mf}{13.605} \PY{c+c1}{\PYZsh{} [eV], Rydbergenergie}
         \PY{n}{h} \PY{o}{=} \PY{l+m+mf}{4.13567e\PYZhy{}15} \PY{c+c1}{\PYZsh{} [eV s], Plancksches Wirkungsquantum}
         \PY{n}{c} \PY{o}{=} \PY{l+m+mf}{2.9979e8} \PY{c+c1}{\PYZsh{} [m/s], Lichtgeschwindigkeit}
\end{Verbatim}


    \hypertarget{nebenserie-md-to-3p}{%
\subsubsection{\texorpdfstring{1. Nebenserie
(\(md \to 3p\))}{1. Nebenserie (md \textbackslash{}to 3p)}}\label{nebenserie-md-to-3p}}

    Wir nehmen an, dass die Linie bei 818,8 nm zu m = 3 gehört. Daraus
ergibt sich:

    \begin{Verbatim}[commandchars=\\\{\}]
{\color{incolor}In [{\color{incolor}44}]:} \PY{n}{E\PYZus{}3p} \PY{o}{=} \PY{n}{E\PYZus{}Ry}\PY{o}{/}\PY{l+m+mi}{3}\PY{o}{*}\PY{o}{*}\PY{l+m+mi}{2} \PY{o}{\PYZhy{}} \PY{n}{h}\PY{o}{*}\PY{n}{c}\PY{o}{/}\PY{l+m+mf}{818.8e\PYZhy{}9}
         \PY{n}{E\PYZus{}3p\PYZus{}} \PY{o}{=} \PY{n}{E\PYZus{}Ry}\PY{o}{/}\PY{l+m+mi}{3}\PY{o}{*}\PY{o}{*}\PY{l+m+mi}{2} \PY{o}{\PYZhy{}} \PY{n}{h}\PY{o}{*}\PY{n}{c}\PY{o}{/}\PY{p}{(}\PY{p}{(}\PY{l+m+mf}{818.8} \PY{o}{+} \PY{l+m+mf}{1.4}\PY{p}{)}\PY{o}{*}\PY{l+m+mf}{1e\PYZhy{}9}\PY{p}{)}
         \PY{n+nb}{print}\PY{p}{(}\PY{l+s+s2}{\PYZdq{}}\PY{l+s+s2}{E\PYZus{}3p = }\PY{l+s+s2}{\PYZdq{}}\PY{p}{,} \PY{n}{E\PYZus{}3p}\PY{p}{,} \PY{l+s+s2}{\PYZdq{}}\PY{l+s+s2}{+\PYZhy{}}\PY{l+s+s2}{\PYZdq{}}\PY{p}{,} \PY{n+nb}{abs}\PY{p}{(}\PY{n}{E\PYZus{}3p} \PY{o}{\PYZhy{}} \PY{n}{E\PYZus{}3p\PYZus{}}\PY{p}{)}\PY{p}{)}
\end{Verbatim}


    \begin{Verbatim}[commandchars=\\\{\}]
E\_3p =  -3.0258734440237745 +- 0.0025846006928795795

    \end{Verbatim}

    Der Fehler wird durch Einsetzen einer Grenze des Fehlerbereichs
berechnet.

    Nun können die erwarteten Wellenlängen berechnet werden, wobei sich die
Fehler wieder durch Einsetzen ergeben:

    \begin{Verbatim}[commandchars=\\\{\}]
{\color{incolor}In [{\color{incolor}40}]:} \PY{k}{for} \PY{n}{m} \PY{o+ow}{in} \PY{n+nb}{range}\PY{p}{(}\PY{l+m+mi}{3}\PY{p}{,} \PY{l+m+mi}{13}\PY{p}{)}\PY{p}{:}
             \PY{n}{l} \PY{o}{=} \PY{n}{h}\PY{o}{*}\PY{n}{c}\PY{o}{/}\PY{p}{(}\PY{n}{E\PYZus{}Ry}\PY{o}{/}\PY{n}{m}\PY{o}{*}\PY{o}{*}\PY{l+m+mi}{2} \PY{o}{\PYZhy{}} \PY{n}{E\PYZus{}3p}\PY{p}{)}\PY{o}{*}\PY{l+m+mf}{1e9}
             \PY{n}{l\PYZus{}} \PY{o}{=} \PY{n}{h}\PY{o}{*}\PY{n}{c}\PY{o}{/}\PY{p}{(}\PY{n}{E\PYZus{}Ry}\PY{o}{/}\PY{n}{m}\PY{o}{*}\PY{o}{*}\PY{l+m+mi}{2} \PY{o}{\PYZhy{}} \PY{n}{E\PYZus{}3p\PYZus{}}\PY{p}{)}\PY{o}{*}\PY{l+m+mf}{1e9}
             \PY{n}{Dl} \PY{o}{=} \PY{n+nb}{abs}\PY{p}{(}\PY{n}{l} \PY{o}{\PYZhy{}} \PY{n}{l\PYZus{}}\PY{p}{)}
             \PY{n+nb}{print}\PY{p}{(}\PY{l+s+s1}{\PYZsq{}}\PY{l+s+s1}{m=}\PY{l+s+si}{\PYZob{}m:2d\PYZcb{}}\PY{l+s+s1}{, lambda=}\PY{l+s+si}{\PYZob{}l:6.2f\PYZcb{}}\PY{l+s+s1}{+\PYZhy{}}\PY{l+s+si}{\PYZob{}Dl:6.2f\PYZcb{}}\PY{l+s+s1}{\PYZsq{}}\PY{o}{.}\PY{n}{format}\PY{p}{(}\PY{n}{m}\PY{o}{=}\PY{n}{m}\PY{p}{,} \PY{n}{l}\PY{o}{=}\PY{n}{l}\PY{p}{,} \PY{n}{Dl}\PY{o}{=}\PY{n}{Dl}\PY{p}{)}\PY{p}{)}
\end{Verbatim}


    \begin{Verbatim}[commandchars=\\\{\}]
m= 3, lambda=818.80+-  1.40
m= 4, lambda=569.89+-  0.68
m= 5, lambda=499.60+-  0.52
m= 6, lambda=468.22+-  0.46
m= 7, lambda=451.14+-  0.42
m= 8, lambda=440.70+-  0.41
m= 9, lambda=433.82+-  0.39
m=10, lambda=429.03+-  0.38
m=11, lambda=425.56+-  0.38
m=12, lambda=422.95+-  0.37

    \end{Verbatim}

    \hypertarget{nebenserie-ms-to-3p}{%
\subsubsection{\texorpdfstring{2. Nebenserie
(\(ms \to 3p\))}{2. Nebenserie (ms \textbackslash{}to 3p)}}\label{nebenserie-ms-to-3p}}

    Genau analog zur 1.Nebenserie (nur, dass diesmal der Korrekturterm nicht
vernachlässigt wird):

    \begin{Verbatim}[commandchars=\\\{\}]
{\color{incolor}In [{\color{incolor}45}]:} \PY{n}{E\PYZus{}3s} \PY{o}{=} \PY{n}{E\PYZus{}3p} \PY{o}{\PYZhy{}} \PY{n}{h}\PY{o}{*}\PY{n}{c}\PY{o}{/}\PY{l+m+mf}{588.8e\PYZhy{}9}
         \PY{n}{E\PYZus{}3s\PYZus{}} \PY{o}{=} \PY{n}{E\PYZus{}3p\PYZus{}} \PY{o}{\PYZhy{}} \PY{n}{h}\PY{o}{*}\PY{n}{c}\PY{o}{/}\PY{p}{(}\PY{p}{(}\PY{l+m+mf}{588.8} \PY{o}{+} \PY{l+m+mf}{1.0}\PY{p}{)}\PY{o}{*}\PY{l+m+mf}{1e\PYZhy{}9}\PY{p}{)}
         \PY{n+nb}{print}\PY{p}{(}\PY{l+s+s2}{\PYZdq{}}\PY{l+s+s2}{E\PYZus{}3s = }\PY{l+s+s2}{\PYZdq{}}\PY{p}{,} \PY{n}{E\PYZus{}3s}\PY{p}{,} \PY{l+s+s2}{\PYZdq{}}\PY{l+s+s2}{+\PYZhy{}}\PY{l+s+s2}{\PYZdq{}}\PY{p}{,} \PY{n+nb}{abs}\PY{p}{(}\PY{n}{E\PYZus{}3s} \PY{o}{\PYZhy{}} \PY{n}{E\PYZus{}3s\PYZus{}}\PY{p}{)}\PY{p}{)}
\end{Verbatim}


    \begin{Verbatim}[commandchars=\\\{\}]
E\_3s =  -5.1315672437860025 +- 0.006154783466297609

    \end{Verbatim}

    \begin{Verbatim}[commandchars=\\\{\}]
{\color{incolor}In [{\color{incolor}46}]:} \PY{n}{Delta\PYZus{}s} \PY{o}{=} \PY{n}{sqrt}\PY{p}{(}\PY{n}{E\PYZus{}Ry}\PY{o}{/}\PY{n}{E\PYZus{}3s}\PY{p}{)} \PY{o}{\PYZhy{}} \PY{l+m+mi}{3}
         \PY{n}{Delta\PYZus{}s\PYZus{}} \PY{o}{=} \PY{n}{sqrt}\PY{p}{(}\PY{n}{E\PYZus{}Ry}\PY{o}{/}\PY{n}{E\PYZus{}3s\PYZus{}}\PY{p}{)} \PY{o}{\PYZhy{}} \PY{l+m+mi}{3}
         \PY{n+nb}{print}\PY{p}{(}\PY{l+s+s2}{\PYZdq{}}\PY{l+s+s2}{Delta\PYZus{}s = }\PY{l+s+s2}{\PYZdq{}}\PY{p}{,} \PY{n}{Delta\PYZus{}s}\PY{p}{,} \PY{l+s+s2}{\PYZdq{}}\PY{l+s+s2}{+\PYZhy{}}\PY{l+s+s2}{\PYZdq{}}\PY{p}{,} \PY{n+nb}{abs}\PY{p}{(}\PY{n}{Delta\PYZus{}s} \PY{o}{\PYZhy{}} \PY{n}{Delta\PYZus{}s\PYZus{}}\PY{p}{)}\PY{p}{)}
\end{Verbatim}


    \begin{Verbatim}[commandchars=\\\{\}]
Delta\_s =  -1.371738099720573 +- 0.000977345016534903

    \end{Verbatim}

    \begin{Verbatim}[commandchars=\\\{\}]
{\color{incolor}In [{\color{incolor}47}]:} \PY{k}{for} \PY{n}{m} \PY{o+ow}{in} \PY{n+nb}{range}\PY{p}{(}\PY{l+m+mi}{4}\PY{p}{,} \PY{l+m+mi}{10}\PY{p}{)}\PY{p}{:}
             \PY{n}{l} \PY{o}{=} \PY{n}{h}\PY{o}{*}\PY{n}{c}\PY{o}{/}\PY{p}{(}\PY{n}{E\PYZus{}Ry}\PY{o}{/}\PY{p}{(}\PY{n}{m} \PY{o}{\PYZhy{}} \PY{n}{Delta\PYZus{}s}\PY{p}{)}\PY{o}{*}\PY{o}{*}\PY{l+m+mi}{2} \PY{o}{\PYZhy{}} \PY{n}{E\PYZus{}3p}\PY{p}{)}\PY{o}{*}\PY{l+m+mf}{1e9}
             \PY{n}{l\PYZus{}} \PY{o}{=} \PY{n}{h}\PY{o}{*}\PY{n}{c}\PY{o}{/}\PY{p}{(}\PY{n}{E\PYZus{}Ry}\PY{o}{/}\PY{p}{(}\PY{n}{m} \PY{o}{\PYZhy{}} \PY{n}{Delta\PYZus{}s\PYZus{}}\PY{p}{)}\PY{o}{*}\PY{o}{*}\PY{l+m+mi}{2} \PY{o}{\PYZhy{}} \PY{n}{E\PYZus{}3p\PYZus{}}\PY{p}{)}\PY{o}{*}\PY{l+m+mf}{1e9}
             \PY{n}{Dl} \PY{o}{=} \PY{n+nb}{abs}\PY{p}{(}\PY{n}{l} \PY{o}{\PYZhy{}} \PY{n}{l\PYZus{}}\PY{p}{)}
             \PY{n+nb}{print}\PY{p}{(}\PY{l+s+s1}{\PYZsq{}}\PY{l+s+s1}{m=}\PY{l+s+si}{\PYZob{}m:2d\PYZcb{}}\PY{l+s+s1}{, lambda=}\PY{l+s+si}{\PYZob{}l:6.2f\PYZcb{}}\PY{l+s+s1}{+\PYZhy{}}\PY{l+s+si}{\PYZob{}Dl:6.2f\PYZcb{}}\PY{l+s+s1}{\PYZsq{}}\PY{o}{.}\PY{n}{format}\PY{p}{(}\PY{n}{m}\PY{o}{=}\PY{n}{m}\PY{p}{,} \PY{n}{l}\PY{o}{=}\PY{n}{l}\PY{p}{,} \PY{n}{Dl}\PY{o}{=}\PY{n}{Dl}\PY{p}{)}\PY{p}{)}
\end{Verbatim}


    \begin{Verbatim}[commandchars=\\\{\}]
m= 4, lambda=485.37+-  0.52
m= 5, lambda=460.77+-  0.46
m= 6, lambda=446.70+-  0.43
m= 7, lambda=437.83+-  0.41
m= 8, lambda=431.85+-  0.39
m= 9, lambda=427.62+-  0.39

    \end{Verbatim}

    \hypertarget{hauptserie-mp-to-3s}{%
\subsubsection{\texorpdfstring{Hauptserie
(\(mp \to 3s\))}{Hauptserie (mp \textbackslash{}to 3s)}}\label{hauptserie-mp-to-3s}}

    \begin{Verbatim}[commandchars=\\\{\}]
{\color{incolor}In [{\color{incolor}48}]:} \PY{n}{Delta\PYZus{}p} \PY{o}{=} \PY{n}{sqrt}\PY{p}{(}\PY{n}{E\PYZus{}Ry}\PY{o}{/}\PY{n}{E\PYZus{}3p}\PY{p}{)} \PY{o}{\PYZhy{}} \PY{l+m+mi}{3}
         \PY{n}{Delta\PYZus{}p\PYZus{}} \PY{o}{=} \PY{n}{sqrt}\PY{p}{(}\PY{n}{E\PYZus{}Ry}\PY{o}{/}\PY{n}{E\PYZus{}3p\PYZus{}}\PY{p}{)} \PY{o}{\PYZhy{}} \PY{l+m+mi}{3}
         \PY{n+nb}{print}\PY{p}{(}\PY{l+s+s2}{\PYZdq{}}\PY{l+s+s2}{Delta\PYZus{}p = }\PY{l+s+s2}{\PYZdq{}}\PY{p}{,} \PY{n}{Delta\PYZus{}p}\PY{p}{,} \PY{l+s+s2}{\PYZdq{}}\PY{l+s+s2}{+\PYZhy{}}\PY{l+s+s2}{\PYZdq{}}\PY{p}{,} \PY{n+nb}{abs}\PY{p}{(}\PY{n}{Delta\PYZus{}p} \PY{o}{\PYZhy{}} \PY{n}{Delta\PYZus{}p\PYZus{}}\PY{p}{)}\PY{p}{)}
\end{Verbatim}


    \begin{Verbatim}[commandchars=\\\{\}]
Delta\_p =  -0.879570229013229 +- 0.0009061809392800768

    \end{Verbatim}

    \begin{Verbatim}[commandchars=\\\{\}]
{\color{incolor}In [{\color{incolor}49}]:} \PY{k}{for} \PY{n}{m} \PY{o+ow}{in} \PY{n+nb}{range}\PY{p}{(}\PY{l+m+mi}{4}\PY{p}{,} \PY{l+m+mi}{10}\PY{p}{)}\PY{p}{:}
             \PY{n}{l} \PY{o}{=} \PY{n}{h}\PY{o}{*}\PY{n}{c}\PY{o}{/}\PY{p}{(}\PY{n}{E\PYZus{}Ry}\PY{o}{/}\PY{p}{(}\PY{n}{m} \PY{o}{\PYZhy{}} \PY{n}{Delta\PYZus{}p}\PY{p}{)}\PY{o}{*}\PY{o}{*}\PY{l+m+mi}{2} \PY{o}{\PYZhy{}} \PY{n}{E\PYZus{}3s}\PY{p}{)}\PY{o}{*}\PY{l+m+mf}{1e9}
             \PY{n}{l\PYZus{}} \PY{o}{=} \PY{n}{h}\PY{o}{*}\PY{n}{c}\PY{o}{/}\PY{p}{(}\PY{n}{E\PYZus{}Ry}\PY{o}{/}\PY{p}{(}\PY{n}{m} \PY{o}{\PYZhy{}} \PY{n}{Delta\PYZus{}p\PYZus{}}\PY{p}{)}\PY{o}{*}\PY{o}{*}\PY{l+m+mi}{2} \PY{o}{\PYZhy{}} \PY{n}{E\PYZus{}3s\PYZus{}}\PY{p}{)}\PY{o}{*}\PY{l+m+mf}{1e9}
             \PY{n}{Dl} \PY{o}{=} \PY{n+nb}{abs}\PY{p}{(}\PY{n}{l} \PY{o}{\PYZhy{}} \PY{n}{l\PYZus{}}\PY{p}{)}
             \PY{n+nb}{print}\PY{p}{(}\PY{l+s+s1}{\PYZsq{}}\PY{l+s+s1}{m=}\PY{l+s+si}{\PYZob{}m:2d\PYZcb{}}\PY{l+s+s1}{, lambda=}\PY{l+s+si}{\PYZob{}l:6.2f\PYZcb{}}\PY{l+s+s1}{+\PYZhy{}}\PY{l+s+si}{\PYZob{}Dl:6.2f\PYZcb{}}\PY{l+s+s1}{\PYZsq{}}\PY{o}{.}\PY{n}{format}\PY{p}{(}\PY{n}{m}\PY{o}{=}\PY{n}{m}\PY{p}{,} \PY{n}{l}\PY{o}{=}\PY{n}{l}\PY{p}{,} \PY{n}{Dl}\PY{o}{=}\PY{n}{Dl}\PY{p}{)}\PY{p}{)}
\end{Verbatim}


    \begin{Verbatim}[commandchars=\\\{\}]
m= 4, lambda=271.88+-  0.38
m= 5, lambda=261.68+-  0.35
m= 6, lambda=255.95+-  0.33
m= 7, lambda=252.39+-  0.32
m= 8, lambda=250.02+-  0.31
m= 9, lambda=248.35+-  0.31

    \end{Verbatim}

    \hypertarget{bestimmung-von-serienenergien-und-korrekturtermen}{%
\subsection{Bestimmung von Serienenergien und
Korrekturtermen}\label{bestimmung-von-serienenergien-und-korrekturtermen}}

    \hypertarget{nebenserie}{%
\subsubsection{1. Nebenserie}\label{nebenserie}}

    Die zugeordneten Wellenlängen sind:

    \begin{Verbatim}[commandchars=\\\{\}]
{\color{incolor}In [{\color{incolor}64}]:} \PY{n}{wellenl} \PY{o}{=} \PY{n}{np}\PY{o}{.}\PY{n}{array}\PY{p}{(}\PY{p}{[}\PY{l+m+mf}{818.8}\PY{p}{,} \PY{l+m+mf}{568.1}\PY{p}{,} \PY{l+m+mf}{497.7}\PY{p}{,} \PY{l+m+mf}{466.3}\PY{p}{,} \PY{l+m+mf}{450.0}\PY{p}{,} \PY{l+m+mf}{438.6}\PY{p}{,} \PY{l+m+mf}{433.0}\PY{p}{,} \PY{l+m+mf}{429.7}\PY{p}{,} \PY{l+m+mf}{426.3}\PY{p}{]}\PY{p}{)}
         \PY{n}{fehler} \PY{o}{=} \PY{n}{np}\PY{o}{.}\PY{n}{array}\PY{p}{(}\PY{p}{[}\PY{l+m+mf}{1.4}\PY{p}{,} \PY{l+m+mf}{1.0}\PY{p}{,} \PY{l+m+mf}{0.7}\PY{p}{,} \PY{l+m+mf}{0.9}\PY{p}{,} \PY{l+m+mf}{2.0}\PY{p}{,} \PY{l+m+mf}{1.0}\PY{p}{,} \PY{l+m+mf}{1.2}\PY{p}{,} \PY{l+m+mf}{1.0}\PY{p}{,} \PY{l+m+mf}{1.3}\PY{p}{]}\PY{p}{)}
         \PY{n}{quantenz} \PY{o}{=} \PY{n}{np}\PY{o}{.}\PY{n}{arange}\PY{p}{(}\PY{l+m+mi}{3}\PY{p}{,} \PY{l+m+mi}{12}\PY{p}{)}
\end{Verbatim}


    Die folgende Funktion, die die Wellenlänge angibt, soll an diese Daten
gefittet werden:

    \begin{Verbatim}[commandchars=\\\{\}]
{\color{incolor}In [{\color{incolor}65}]:} \PY{k}{def} \PY{n+nf}{fit\PYZus{}func\PYZus{}1}\PY{p}{(}\PY{n}{m}\PY{p}{,} \PY{n}{\PYZus{}E\PYZus{}Ry}\PY{p}{,} \PY{n}{\PYZus{}E\PYZus{}3p}\PY{p}{,} \PY{n}{D\PYZus{}d}\PY{p}{)}\PY{p}{:}
             \PY{k}{return} \PY{n}{h}\PY{o}{*}\PY{n}{c}\PY{o}{/}\PY{p}{(}\PY{n}{\PYZus{}E\PYZus{}Ry}\PY{o}{/}\PY{p}{(}\PY{n}{m} \PY{o}{\PYZhy{}} \PY{n}{D\PYZus{}d}\PY{p}{)}\PY{o}{*}\PY{o}{*}\PY{l+m+mi}{2} \PY{o}{\PYZhy{}} \PY{n}{\PYZus{}E\PYZus{}3p}\PY{p}{)}\PY{o}{*}\PY{l+m+mf}{1e9}
\end{Verbatim}


    Dazu werden Startwerte für die Parameter \(E_{Ry}\), \(E_{3p}\) und
\(\Delta_d\) gewählt:

    \begin{Verbatim}[commandchars=\\\{\}]
{\color{incolor}In [{\color{incolor}66}]:} \PY{n}{para} \PY{o}{=} \PY{p}{[}\PY{o}{\PYZhy{}}\PY{l+m+mf}{13.6}\PY{p}{,} \PY{o}{\PYZhy{}}\PY{l+m+mi}{3}\PY{p}{,} \PY{o}{\PYZhy{}}\PY{l+m+mf}{0.02}\PY{p}{]}
         \PY{n}{popt}\PY{p}{,} \PY{n}{pcov} \PY{o}{=} \PY{n}{curve\PYZus{}fit}\PY{p}{(}\PY{n}{fit\PYZus{}func\PYZus{}1}\PY{p}{,} \PY{n}{quantenz}\PY{p}{,} \PY{n}{wellenl}\PY{p}{,} \PY{n}{sigma} \PY{o}{=} \PY{n}{fehler}\PY{p}{,} \PY{n}{p0} \PY{o}{=} \PY{n}{para}\PY{p}{)}
\end{Verbatim}


    Nun werden die Ergebnisse des Fits mit den jeweiligen Fehlern
ausgegeben:

    \begin{Verbatim}[commandchars=\\\{\}]
{\color{incolor}In [{\color{incolor}67}]:} \PY{n+nb}{print}\PY{p}{(}\PY{l+s+s2}{\PYZdq{}}\PY{l+s+s2}{E\PYZus{}Ry = }\PY{l+s+s2}{\PYZdq{}}\PY{p}{,} \PY{n}{popt}\PY{p}{[}\PY{l+m+mi}{0}\PY{p}{]}\PY{p}{,} \PY{l+s+s2}{\PYZdq{}}\PY{l+s+s2}{, Standardfehler = }\PY{l+s+s2}{\PYZdq{}}\PY{p}{,} \PY{n}{np}\PY{o}{.}\PY{n}{sqrt}\PY{p}{(}\PY{n}{pcov}\PY{p}{[}\PY{l+m+mi}{0}\PY{p}{]}\PY{p}{[}\PY{l+m+mi}{0}\PY{p}{]}\PY{p}{)}\PY{p}{)}
         \PY{n+nb}{print}\PY{p}{(}\PY{l+s+s2}{\PYZdq{}}\PY{l+s+s2}{E\PYZus{}3p = }\PY{l+s+s2}{\PYZdq{}}\PY{p}{,} \PY{n}{popt}\PY{p}{[}\PY{l+m+mi}{1}\PY{p}{]}\PY{p}{,} \PY{l+s+s2}{\PYZdq{}}\PY{l+s+s2}{, Standardfehler = }\PY{l+s+s2}{\PYZdq{}}\PY{p}{,} \PY{n}{np}\PY{o}{.}\PY{n}{sqrt}\PY{p}{(}\PY{n}{pcov}\PY{p}{[}\PY{l+m+mi}{1}\PY{p}{]}\PY{p}{[}\PY{l+m+mi}{1}\PY{p}{]}\PY{p}{)}\PY{p}{)}
         \PY{n+nb}{print}\PY{p}{(}\PY{l+s+s2}{\PYZdq{}}\PY{l+s+s2}{D\PYZus{}d = }\PY{l+s+s2}{\PYZdq{}}\PY{p}{,} \PY{n}{popt}\PY{p}{[}\PY{l+m+mi}{2}\PY{p}{]}\PY{p}{,} \PY{l+s+s2}{\PYZdq{}}\PY{l+s+s2}{, Standardfehler = }\PY{l+s+s2}{\PYZdq{}}\PY{p}{,} \PY{n}{np}\PY{o}{.}\PY{n}{sqrt}\PY{p}{(}\PY{n}{pcov}\PY{p}{[}\PY{l+m+mi}{2}\PY{p}{]}\PY{p}{[}\PY{l+m+mi}{2}\PY{p}{]}\PY{p}{)}\PY{p}{)}
\end{Verbatim}


    \begin{Verbatim}[commandchars=\\\{\}]
E\_Ry =  -12.966162642316942 , Standardfehler =  0.2995829353625529
E\_3p =  -3.024395644570556 , Standardfehler =  0.005166940588024771
D\_d =  0.07012734717231944 , Standardfehler =  0.02991574849036398

    \end{Verbatim}

    Schließlich werden noch die \(\chi^2\)- und
\(\chi_{\text{red}}^2\)-Werte berechnet:

    \begin{Verbatim}[commandchars=\\\{\}]
{\color{incolor}In [{\color{incolor}74}]:} \PY{n}{chi2\PYZus{}} \PY{o}{=} \PY{n}{np}\PY{o}{.}\PY{n}{sum}\PY{p}{(}\PY{p}{(}\PY{n}{fit\PYZus{}func\PYZus{}1}\PY{p}{(}\PY{n}{quantenz}\PY{p}{,} \PY{o}{*}\PY{n}{popt}\PY{p}{)} \PY{o}{\PYZhy{}} \PY{n}{wellenl}\PY{p}{)}\PY{o}{*}\PY{o}{*}\PY{l+m+mi}{2}\PY{o}{/}\PY{n}{fehler}\PY{o}{*}\PY{o}{*}\PY{l+m+mi}{2}\PY{p}{)}
         \PY{n}{dof} \PY{o}{=} \PY{n+nb}{len}\PY{p}{(}\PY{n}{quantenz}\PY{p}{)} \PY{o}{\PYZhy{}} \PY{l+m+mi}{3}
         \PY{n}{chi2\PYZus{}red} \PY{o}{=} \PY{n}{chi2\PYZus{}}\PY{o}{/}\PY{n}{dof}
         \PY{n+nb}{print}\PY{p}{(}\PY{l+s+s2}{\PYZdq{}}\PY{l+s+s2}{chi2 = }\PY{l+s+s2}{\PYZdq{}}\PY{p}{,} \PY{n}{chi2\PYZus{}}\PY{p}{)}
         \PY{n+nb}{print}\PY{p}{(}\PY{l+s+s2}{\PYZdq{}}\PY{l+s+s2}{chi2\PYZus{}red = }\PY{l+s+s2}{\PYZdq{}}\PY{p}{,} \PY{n}{chi2\PYZus{}red}\PY{p}{)}
\end{Verbatim}


    \begin{Verbatim}[commandchars=\\\{\}]
chi2 =  4.605958153596782
chi2\_red =  0.7676596922661303

    \end{Verbatim}

    \begin{Verbatim}[commandchars=\\\{\}]
{\color{incolor}In [{\color{incolor}75}]:} \PY{n}{prob} \PY{o}{=} \PY{n+nb}{round}\PY{p}{(}\PY{l+m+mi}{1} \PY{o}{\PYZhy{}} \PY{n}{chi2}\PY{o}{.}\PY{n}{cdf}\PY{p}{(}\PY{n}{chi2\PYZus{}}\PY{p}{,} \PY{n}{dof}\PY{p}{)}\PY{p}{,} \PY{l+m+mi}{2}\PY{p}{)}\PY{o}{*}\PY{l+m+mi}{100}
         \PY{n+nb}{print}\PY{p}{(}\PY{l+s+s2}{\PYZdq{}}\PY{l+s+s2}{Wahrscheinlichkeit:}\PY{l+s+s2}{\PYZdq{}}\PY{p}{,} \PY{n}{prob}\PY{p}{,} \PY{l+s+s2}{\PYZdq{}}\PY{l+s+s2}{\PYZpc{}}\PY{l+s+s2}{\PYZdq{}}\PY{p}{)}
\end{Verbatim}


    \begin{Verbatim}[commandchars=\\\{\}]
Wahrscheinlichkeit: 60.0 \%

    \end{Verbatim}

    Die zugeordneten Wellenlängen werden gemeinsam mit der gefitteten
Funktion in ein Diagramm eingetragen:

    \begin{Verbatim}[commandchars=\\\{\}]
{\color{incolor}In [{\color{incolor}70}]:} \PY{n}{plt}\PY{o}{.}\PY{n}{errorbar}\PY{p}{(}\PY{n}{quantenz}\PY{p}{,} \PY{n}{wellenl}\PY{p}{,} \PY{n}{fehler}\PY{p}{,} \PY{n}{fmt} \PY{o}{=} \PY{l+s+s2}{\PYZdq{}}\PY{l+s+s2}{.}\PY{l+s+s2}{\PYZdq{}}\PY{p}{)}
         \PY{n}{plt}\PY{o}{.}\PY{n}{xlabel}\PY{p}{(}\PY{l+s+s2}{\PYZdq{}}\PY{l+s+s2}{Quantenzahl}\PY{l+s+s2}{\PYZdq{}}\PY{p}{)}
         \PY{n}{plt}\PY{o}{.}\PY{n}{ylabel}\PY{p}{(}\PY{l+s+s2}{\PYZdq{}}\PY{l+s+s2}{Wellenlänge [nm]}\PY{l+s+s2}{\PYZdq{}}\PY{p}{)}
         \PY{n}{plt}\PY{o}{.}\PY{n}{title}\PY{p}{(}\PY{l+s+s2}{\PYZdq{}}\PY{l+s+s2}{1. Nebenserie des Na\PYZhy{}Atoms}\PY{l+s+s2}{\PYZdq{}}\PY{p}{)}
         \PY{n}{x} \PY{o}{=} \PY{n}{np}\PY{o}{.}\PY{n}{linspace}\PY{p}{(}\PY{l+m+mf}{2.8}\PY{p}{,} \PY{l+m+mf}{12.2}\PY{p}{,} \PY{l+m+mi}{100}\PY{p}{)}
         \PY{n}{plt}\PY{o}{.}\PY{n}{plot}\PY{p}{(}\PY{n}{x}\PY{p}{,} \PY{n}{fit\PYZus{}func\PYZus{}1}\PY{p}{(}\PY{n}{x}\PY{p}{,} \PY{o}{*}\PY{n}{popt}\PY{p}{)}\PY{p}{)}
         \PY{n}{plt}\PY{o}{.}\PY{n}{savefig}\PY{p}{(}\PY{l+s+s2}{\PYZdq{}}\PY{l+s+s2}{figures/nebenserie\PYZus{}1.pdf}\PY{l+s+s2}{\PYZdq{}}\PY{p}{,} \PY{n+nb}{format} \PY{o}{=} \PY{l+s+s2}{\PYZdq{}}\PY{l+s+s2}{pdf}\PY{l+s+s2}{\PYZdq{}}\PY{p}{)}
\end{Verbatim}


    \begin{center}
    \adjustimage{max size={0.9\linewidth}{0.9\paperheight}}{output_59_0.png}
    \end{center}
    { \hspace*{\fill} \\}
    

    % Add a bibliography block to the postdoc
    
    
    
    \end{document}
